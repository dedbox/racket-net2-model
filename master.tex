\documentclass[letterpaper,12pt]{article}

\usepackage[margin=1in]{geometry}
\usepackage{amsmath,amssymb}
\usepackage{semantic}
\usepackage{mathpartir}

\pagestyle{empty}

% sets
\DeclareMathOperator{\Bs}{\ensuremath{\mathcal{B}}}
\DeclareMathOperator{\Ls}{\ensuremath{\mathcal{L}}}
\DeclareMathOperator{\PIs}{\ensuremath{\mathcal{P_I}}}
\DeclareMathOperator{\POs}{\ensuremath{\mathcal{P_O}}}
\DeclareMathOperator{\Ts}{\ensuremath{\mathcal{T}}}
\DeclareMathOperator{\Us}{\ensuremath{\mathcal{U}}}
\DeclareMathOperator{\Void}{\ensuremath{\varnothing}}

% relations
\DeclareMathOperator{\Accept}{accept}
\DeclareMathOperator{\Connect}{connect}
\DeclareMathOperator{\Listen}{listen}
\DeclareMathOperator{\Receive}{receive}
\DeclareMathOperator{\Release}{release}
\DeclareMathOperator{\Send}{send}

% rules
\newcommand{\Rule}[1]{\text{\textsc{#1}}}
\newcommand{\Infer}[3][]{\inferrule{#2}{#3}~\Rule{#1}}

% literals
\newcommand{\Lt}{\text{\texttt{L}}}

% variables
\newcommand{\Lx}{\ensuremath{\Lambda}}
\newcommand{\Gx}{\ensuremath{\Gamma}}
\newcommand{\Px}{\ensuremath{\Pi}}

% helpers
\DeclareMathOperator{\Accepter}{accepter}
\DeclareMathOperator{\Connector}{connector}
\DeclareMathOperator{\Listener}{listener}
\DeclareMathOperator{\Receiver}{receiver}
\DeclareMathOperator{\Sender}{sender}

% symbols
\mathlig{|-->}{\mapsto}
\mathlig{-->}{\leadsto}
\mathlig{=>}{\Downarrow}
\mathlig{->}{\rightarrow}
\mathlig{|-}{\vdash}
\mathlig{**}{\times}
\mathlig{<}{\langle}
\mathlig{>}{\rangle}

\begin{document}

\section*{Primary Constructs}

\subsection*{Data Types}

\begin{mathpar}
  \Bs = \text{byte arrays}
  \and \Ls = \text{listeners}
  \and \PIs = \text{input ports}
  \and \POs = \text{output ports}
  \and \Ts = \text{transports}
  \and \Us = \text{URIs}
  \and \Void = \text{void}
\end{mathpar}

A \emph{reference} is an opaque token that identifies an external resource.
Listeners and transports are references. A \emph{port} is a reliable,
buffered, ordered byte stream. Ports either consume or produce bytes. A
\emph{URI} is a universal resource identifier as defined in RFC 3986,
restricted here to URL authorities as defined in the roadmap.

\subsection*{API Functions}

\begin{mathpar}
  \Listen : \Us -> \Ls
  \and \Accept : \Ls -> \Ts
  \and \Connect : \Us -> \Ts
  \and \Release : \Ls \cup \Ts -> \Void
  \and \Send : \Ts ** \Bs -> \Void
  \and \Receive : \Ts -> \Bs
\end{mathpar}

\subsection*{Internal Functions}

\begin{mathpar}
  \Listener : \Us -> \Ls
  \and \Accepter : \Ls -> \Ts ** \Us ** \PIs ** \POs
  \and \Connector : \Us -> \Ts ** \Us ** \PIs ** \POs
  \\
  \Sender : \Bs ** \POs -> \POs
  \and \Receiver : \PIs -> \Bs ** \PIs
\end{mathpar}

An \emph{internal function} is a placeholder for implementation details that
do not affect the API.

\subsection*{Run-time Environment}

\begin{mathpar}
  \Lambda : \Ls -> \Us
  \and \Gamma : \Ts -> \Us ** \Us
  \and \Pi : \Ts -> \PIs ** \POs
\end{mathpar}

The run-time state consists of three maps. Adding a binding to a map tells the
run-time to perform a computation with visible effects until the binding is
removed. In $\Lx$, $\cdot |--> u_L$ listens for connections on $u_L$. In
$\Gx$, $\cdot |--> (u_L,u_R)$ establishes a connection between $u_L$ and
$u_R$. In $\Px$, $\cdot |--> (p_I,p_O)$ opens ports $p_I$ and $p_O$ for
reading and writing. When $\Px(t) = (p_I,p_O)$, re-binding $t |--> (p_I',p_O)$
or $t |--> (p_I,p_O')$ tells the run-time to read or write bytes from or to
the port.

\subsection*{Operational Semantics}

\begin{mathpar}
  \Infer[Lsn]{
    \Listener(u_L) = \ell
  }{
    \Lx, \Gx, \Px |- \Listen(u_L) --> [ \ell |--> u_L ] \Lx, \Gx, \Px |- \ell
  }

  \Infer[Acc]{
    \Lx(\ell) = u_L
    \\ \Accepter(\ell) = (t, u_R, p_I, p_O)
  }{
    \Lx, \Gx, \Px |- \Accept(\ell) --> \Lx, [ t |--> (u_L, u_R) ] \Gx, [ t |--> (p_I, p_O) ] \Px |- t
  }

  \Infer[Con]{
    \Connector(u_R) = (t, u_L, p_I, p_O)
  }{
    \Lx, \Gx, \Px |- \Connect(u_R) --> \Lx, [ t |--> (u_L, u_R) ] \Gx, [ t |--> (p_I, p_O) ] \Px |- t
  }

  \Infer[RlsL]{
  }{
    \Lx, \Gx, \Px |- \Release(\ell) --> \Lx \setminus \{ \ell |--> \cdot \}, \Gx, \Px |- \Void
  }

  \Infer[RlsT]{
  }{
    \Lx, \Gx, \Px |- \Release(t) --> \Lx, \Gx \setminus \{ t |--> \cdot \}, \Px \setminus \{ t |--> \cdot \} |- \Void
  }

  \Infer[Snd]{
    \Px(t) = (p_I, p_O)
    \\ \Sender(b, p_O) = p_O'
  }{
    \Lx, \Gx, \Px |- \Send(t, b) --> \Lx, \Gx, [ t |--> (p_I, p_O') ] \Px |- \Void
  }

  \Infer[Rcv]{
    \Px(t) = (p_I, p_O)
    \\ \Receiver(p_I) = (b, p_I')
  }{
    \Lx, \Gx, \Px |- \Receive(t) --> \Lx, \Gx, [ t |--> (p_I', p_O) ] \Px |- b
  }
\end{mathpar}

\Rule{Lsn} says $\Listen(u_L)$ produces a listener $\ell$ on URL authority
$u_L$. \Rule{Acc} says $\Accept(\ell)$ produces a transport $t$ that, when URL
authority $u_R$ connects to $u_L$, represents the connection between $u_L$ and
$u_R$. \Rule{Con} says $\Connect(u_R)$ produces a transport $t$ that, when
$u_R$ accepts the connection from a ``chosen'' URL authority $u_L$, represents
the connection between $u_L$ and $u_R$. \Rule{Snd} says $\Send(t, b)$ writes
the bytes in $b$ to the output port bound to $t$. \Rule{Rcv} says
$\Receive(t)$ produces all available bytes $b$ from the input port bound to
$t$.

\end{document}
